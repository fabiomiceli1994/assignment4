\documentclass[11pt]{article}
%\usepackage[version=3]{mhchem} % Package for chemical equation typesetting
%\usepackage{siunitx} % Provides the \SI{}{} and \si{} command for typesetting SI units
%\usepackage{graphicx} % Required for the inclusion of images
%\usepackage{natbib} % Required to change bibliography style to APA
\usepackage{amsmath} % Required for some math elements 
\usepackage[margin=1in]{geometry}
\usepackage{microtype}
\usepackage[english]{babel}
\usepackage[utf8]{inputenx}
\usepackage{float}
\usepackage{graphics}
\usepackage{graphicx}
\usepackage{subfigure}
\usepackage{amsmath}
\usepackage{textcomp}
\usepackage{makeidx}
\usepackage{hyperref}
\usepackage{braket}
%\usepackage[latin1]{inputenc}
\usepackage{amsthm}
\usepackage{amsfonts}
\usepackage{amssymb}
\usepackage{graphicx}
\usepackage{fullpage}
\usepackage{hyphenat}
\usepackage{float}
\usepackage{longtable}
\usepackage{picture}
\usepackage{multicol}
\usepackage{fancyhdr}
\usepackage{wrapfig}
\usepackage{geometry}
\usepackage{listings}
\usepackage{color}
\definecolor{dkgreen}{rgb}{0,0.6,0}
\definecolor{gray}{rgb}{0.5,0.5,0.5}
\definecolor{mauve}{rgb}{0.58,0,0.82}

\lstset{frame=tb,
	language=C++,
	aboveskip=3mm,
	belowskip=3mm,
	showstringspaces=false,
	columns=flexible,
	basicstyle={\small\ttfamily},
	numbers=none,
	numberstyle=\tiny\color{gray},
	keywordstyle=\color{blue},
	commentstyle=\color{dkgreen},
	stringstyle=\color{mauve},
	breaklines=true,
	breakatwhitespace=true,
	tabsize=3
}

\theoremstyle{theorem}
\newtheorem{theorem}{Theorem}
\newtheorem{proposition}{Proposition}

\theoremstyle{definition}
\newtheorem{definition}{Definition}



%\usepackage{tikz}
%\usetikzlibrary{graphs}
%\usepackage{pgfplots}
%\pgfplotsset{compat=newest}

\setlength\parindent{0pt} % Removes all indentation from paragraphs

\renewcommand{\figurename}{Grafico}
\renewcommand{\tablename}{Tabella}
\newcommand{\at}[2][]{#1\Big|_{#2}}
\renewcommand{\labelenumi}{\alph{enumi}.} % Make numbering in the enumerate environment by letter rather than number (e.g. section 6)

%\usepackage{times} % Uncomment to use the Times New Roman font

%----------------------------------------------------------------------------------------
%	DOCUMENT INFORMATION
%----------------------------------------------------------------------------------------

\title{C1 - Assignment 4 Report: Parabolic Initial-Boundary Value Problem} % Title

\author{Student Number: 1894945} % Author name

\date{\today} % Date for the report

\begin{document}

\maketitle % Insert the title, author and date

\begin{center}
C1 - Assignment 4 Report \hfill
Student Number: 1894945
\vspace{3pt} \hrule \vspace{3pt} \hrule
\end{center}

%\clearpage
\tableofcontents

\clearpage
% If you wish to include an abstract, uncomment the lines below
%\begin{abstract}


%\end{abstract}
%\clearpage 

\section{Introduction}
Intuitively, (Parabolic) Initial-Boundary Value Problems (IVBPs) combine an initial value problem in a distinguished “time-like” direction with a boundary value
problem in “space-like” directions. In this assignment, the approximation of solutions to IVBPs for scalar parabolic differential equations (PDEs) will be considered.

\begin{definition}
	\label{def:IVBP}
	Let $I\subset\mathbb{R}$ be an interval, $\Omega\subset\mathbb{R^d}$ be a simply connected, bounded domain $t_0\in I$, $\mathcal{L}$ a spatial differential operator of the form $\mathcal{L}=-\Delta+\mathbf{p}\cdot\nabla+q$, for $\mathbf{p}$ and $q$ smooth functions. The parabolic initial boundary-value problem (IBVP) is to find $u(\mathbf{x},t):\Omega\times I\to\mathbb{R}$ for which:
	\begin{align}
		\label{eqn:IVBP}
		\begin{cases}
		\partial_t u+\mathcal{L}u=f &\text{for}\hspace{1mm}\mathbf{x}\in\Omega,\hspace{2mm}\forall t\in\ I\\
		u=v &\text{for}\hspace{1mm}\mathbf{x}\in\partial\Omega,\hspace{2mm}\forall t\in I\\
		u(\mathbf{x},0)=u_0(\mathbf{x}) &\text{for}\hspace{1mm}\mathbf{x}\in\Omega
		\end{cases}
	\end{align} 
	with data $f(\mathbf{x},t):\Omega\times I\to\mathbb{R}$, $v(\mathbf{x},t):\partial\Omega\times I\to\mathbb{R}$, $u_0(\mathbf{x}):\Omega\to\mathbb{R}$. Eqn.\eqref{eqn:IVBP} is a parabolic partial differential equation (PDE).\\
\end{definition}
Generally speaking, one wants the operator $\mathcal{L}$ to be uniformly elliptic in the spatial direction for the whole problem to be parabolic .\\

While well posedness for ODEs is satisfactorily answered in a generic way with the Picard-Lindelöf theorem, the same is not true for PDEs, not even linear ones. Often,
existence and uniqueness are established on a case-by-case basis, and in many cases is not known at all (for example famously for the Navier-Stokes equation describing the motion of incompressible fluids). Nevertheless, there is huge interest from an applications perspective to obtain numerical solutions to PDE problems relevant to physics and
engineering. General PDE theory will not be presented here. Instead, methods to combine known methods for IVPs and BVPs to solve problems of the type of problem \ref{def:IVBP} will be discussed. Clearly, the idea is to discretise the spatial direction by replacing $\mathcal{L}$ with an appropriate discretised operator $\mathcal{L}_h$ and using one of the time integration schemes introduced above to treat the time evolution.\\

The following simple IVBP represents the benchmark problem for testing numerical schemes for IVBPs in the present assignment.\\

\begin{definition}
	\label{def:Heat}
	Let $I=[0,T]$, $\Omega=(0,L)$, where $L\in\mathbb{R}^+$ and $T\in\mathbb{R}^+$, and consider the following IVBP:
	\begin{align}
	\label{eqn:Heat}
	\begin{cases}
	\partial_t u - k\partial^2_x u = f &\text{for}\hspace{1mm}x\in\Omega,\hspace{2mm}\forall t\in\ I\\
	u(0, t)=u(1, t)=0 &\text{for}\hspace{1mm}x\in\partial\Omega,\hspace{2mm}\forall t\in I\\
	u(x,0)=u_0(x) &\text{for}\hspace{1mm}x\in\Omega
	\end{cases}
	\end{align} 
	Eqn.\eqref{eqn:Heat} is called the heat equation. Specifically, $u(x,t)$ 	describes the diffusive transport (of matter or heat) in space, with constant diffusivity/conductivity $k\in\mathbb{R}^+$, and where $f(x,t)$ describes
	the local heat production. The heat equation is well posed.\\
\end{definition}

\subsection{Numerical methods to approximate solution to IBVPs}
Two different approaches to finding numerical solution to IBVPs can be adopted. The first one results in the so-called \emph{semi-discretisation schemes}, in which a generic IBVP is first discretised in space and then in time or viceversa. Alternatively, a simultaneous discretisation, leading to a \emph{fully discretised scheme}, can be performed.\\
A detailed presentation of \emph{fully-discretised schemes} can be found in Ref.\cite{numerical-math}. In the present report it will only be pointed out that, although a fully-discretised approach appears to be more convenient, in the case of stiff problems approximated by means of a method characterised by a bounded region of absolute stability, the choice of the discretisation time step and of the mesh size cannot be done arbitrarily, since the two are related by the Courant-Friedrichs-Levy conditions (see \cite{numerical-math}).\\

\subsubsection{Method of lines}
In the present assignment only one of two possible semi-discretisation approaches will be presented. A more general discussion of them can again be found in Ref.\cite{numerical-math} and Ref.\cite{lec-notes}. In particular, only the so called \emph{method of lines} will be presented.\\

Let $ N_x $ and $ N_t $ denote the number of spatial and temporal discretisation points respectively. Also let $ (x_i)_{i=0}^{N_x + 1} $ be a set of points, that discretise $ \Omega = [0,L] $. One has:
\begin{align*}
0 = x_0 < x_1 < \cdots < x_{N_x} < x_{N_x + 1} = L
\end{align*}
Define $ I_k := [x_{k},x_{k+1}] $, with $ |I_k| := h $, for $ k = 0,\ldots,N_x $. The variable h $\in  \mathbb{R}^+$  is the spatial mesh size. Finally, denote by $ u_i : I \to \mathbb{R}$ the approximation of the exact solution $ u $ at $ x_i $ for $ i = 1,\ldots,N_x $.
\\
The method of lines takes the position of first discretising in space. Such a discretisation is here performed through second order central finite differences. In the case of Eqn.\eqref{def:Heat}, the resulting problem becomes a system of ordinary differential equations of the form:

\begin{align*}
	\frac{du_i}{dt} = \kappa \partial^+\partial^- u_i = \kappa\frac{u_{i+1} - 2u_i + u_{i-1}}{h^2},\hspace{4mm} i\in\lbrace 1,\ldots, N_x\rbrace
\end{align*}

Such a problem can be equivalently restated in a convenient matrix form. Denoting by $ \mathbf{u} = (u_1,\ldots,u_{N_x}) : I^{N_x} \to \mathbb{R} $, one has:

\begin{align}\label{Eq:IVP from Method of Lines}
\frac{d\mathbf{u}}{dt} = f(t,\mathbf{u}) = -\frac{\kappa}{h^2}
\begin{pmatrix}
2 & -1 & & & & -1\\
-1 & 2 & -1 & & & \\
& -1 & 2 & -1 & & \\
& & \ddots & \ddots & \ddots & \\
& & & & & -1 \\
-1 & & & -1 & 2
\end{pmatrix}
\mathbf{u} 
\end{align}

Periodic boundary conditions (PBCs) have been additionally imposed on the problem (see \cite{lec-notes}) as required by the assignment instructions\footnote{In Eqn.\eqref{Eq:IVP from Method of Lines} the blank entries represent zeros of tre matrix, which has size $N_x\times N_x$.}.\\
Hence, the semidiscretisation generates an IVP for a system of ODEs which may be solved by known methods for IVPs. In this formalism, one has:


\begin{align}
\label{eqn:lines}
\begin{cases}
	\frac{du_i}{dt}=\partial^+\partial^-u_i+f_i &\forall x_i\in\Omega_h\\
	u_i=0 &\forall x_i\in\Gamma_h
\end{cases}
\end{align} 

with initial conditions $u_i(0)=u_0(x_i)$ for $x_i\in\Omega_h$, $i\in\lbrace 1\ldots N_x\rbrace$. Here $\Omega_h$ represents the set of points in the interior of $\Omega$ and $\Gamma_h$ the set of points on the boundary of $\Omega$, once the spatial discretisation has taken place.\\
To complete the discretisation a solver for ODEs has to be applied to the system. In order to do so, a time-discretisation needs to be performed. Similarly to what has been previously done for the spatial case, let $ (t_n)_{n = 0}^{N_t + 1} $ denote the discretisation of the temporal space $ I = [0,T] $. Hence:

$$0 = t_0 < t_1 < \cdots < t_{N_t} < t_{N_t + 1} = T$$

where $|J_k| = | [t_{k},t_{k+1}] | := \tau\in\mathbb{R}^+$ is the temporal step size, for $ k = 0,\ldots,N_t $. Moreover, denote by $ \mathbf{u}^{i} $ the approximation of the exact solution of the IVP for $ \mathbf{u} $ at $ t_i $ for $ i = 1,\ldots,N_t $.

\subsection{The present case}
In the present assignment we aim to solve Eqn.\eqref{def:Heat} in the case of $L=1$, $T=0.1$, $f\equiv 0$, PBCs and the following initial condition:

\begin{align}
\label{eqn:init-cond}
u_0(x)=
\begin{cases}
0\hspace{4mm}\text{if}\hspace{1mm} x\le\frac{1}{4},\hspace{1mm}x>\frac{3}{4}\\
1\hspace{4mm}\text{otherwise}
\end{cases}
\end{align}

System of equations \eqref{eqn:lines} will be solved by means of Runge-Kutta (RK) methods. The implementation of such methods for the present problem will be now discussed. Set $ f(t_i,\mathbf{u}^i) = f_i $. An s-stage RK method for system \eqref{Eq:IVP from Method of Lines} takes the form:
\begin{align*}
\mathbf{u}^{n+1} = \mathbf{u}^{n} + \tau \mathbf{F}(t_n, \mathbf{u}^n, f, \tau)
\end{align*}
where the increment function $ F $ is defined as
\begin{align*}
\mathbf{F}(t_n, \mathbf{u}^n, f, \tau) &= \sum_{i=1}^{s}b_i\mathbf{K}_i \\
\mathbf{K}_i &= f\left(t_n + c_i\tau, \mathbf{u}^n + \tau\sum_{j=1}^{s}a_{ij}\mathbf{K}_j\right), \quad \text{for } i \in \{1,\ldots,s\}
\end{align*}
and $ A = (a_{ij}) \in \mathbb{R}^{s \times s} $, $ \mathbf{b}, \mathbf{c} \in \mathbb{R}^s $ completely determine the Runge-Kutta method.\\
As it is well known, a specific Runge-Kutta method can be denoted by its \emph{Butcher tableau}, which takes the following form:
\begin{align*}
\renewcommand\arraystretch{1.2}
\begin{array}
{c|c}
\mathbf{c} &
A\\
\hline
& \mathbf{b}
\end{array}.
\end{align*}
The forward Euler method, the backward Euler method and the 3-stage Heun method will be used in this assignment. Their Butcher tableaus are reported here:
\begin{center}
	\begin{minipage}{0.3\linewidth}
		\begin{align*}
		\renewcommand\arraystretch{1.2}
		\textnormal{Forward Euler (FE)} \\
		\begin{array}
		{c|c}
		0 &
		\\
		\hline
		& 1
		\end{array}
		\end{align*}
	\end{minipage}
	\begin{minipage}{0.3\linewidth}
		\begin{align*}
		\renewcommand\arraystretch{1.2}
		\textnormal{Backward Euler (BE)} \\
		\begin{array}
		{c|c}
		1 & 1
		\\
		\hline
		& 1
		\end{array}
		\end{align*}
	\end{minipage}
	\begin{minipage}{0.3\linewidth}
		\begin{align*}
		\renewcommand\arraystretch{1.2}
		\textnormal{3-stage Heun (Heun3)} \\
		\begin{array}
		{c|ccc}
		0 & & \\
		1/3 & 1/3 & \\
		2/3 & 0 & 2/3 \\
		\hline
		& 1/4 & 0 & 3/4
		\end{array}
		\end{align*}
	\end{minipage}
\end{center}

It should now be clear that the numerical implementation of the code is a combination of the techniques proposed for the previous assignments and, in particular, extends the code used in Assignment 3 to a ``vectorial case".\\
It is worth pointing out how such an extension takes place in the case of implicit RK methods, i.e RK methods where at each stage $ K_i $, for a specific $ i \leq s $, $s\in\mathbb{N}$ is determined by solving the implicit equation:

\begin{align}
\label{eqn:K_i}
\mathbf{K}_i = \mathbf{f}\left(t_n + \tau c_i, \mathbf{u}^n + \tau\sum_{j=1}^{i-1}a_{ij}\mathbf{K}_j + \tau a_{ii}\mathbf{K}_i\right)
\end{align}
Thus, at each stage $ i $, one has to solve for $ K_i $ through other means, which in this case coincide with the Newton's method. Using the Newton method, one solves for the root of the function $ \mathbf{g} $ given by:
\begin{align*}
\mathbf{g}(\mathbf{K}_i) := \mathbf{f}\left(t_n + \tau c_i, \mathbf{u}^n + \tau\sum_{j=1}^{i-1}a_{ij}\mathbf{K}_j + \tau a_{ii}\mathbf{K}_i\right) - \mathbf{K}_i
\end{align*}
$\mathbf{K}_i $, which is the root of $g$, is approximated through the following iterative equation:
\begin{align*}
\mathbf{K}_i^{k+1} = \mathbf{K}_i^{k} - (J_\mathbf{g}(\mathbf{K}_{i}^{k}))^{-1}\mathbf{g}(\mathbf{K}_{i}^{k}),
\end{align*}
$ \mathbf{K}_i^k $ being the $ k $th iteration of the approximation of $ \mathbf{K}_i $ and $J_\mathbf{g}(\mathbf{K}_{i}^{k})$ is the Jacobian of $\mathbf{g}$ at the point $\mathbf{K}_i$. We note that to perform the inversion of $(J_\mathbf{g}(\mathbf{K}_{i}^{k})) $, iterative methods have to be used.\\



\section{The program}
\subsection{Implementation}
The program is built on the classes and struct that can be found in the files \emph{vector.hh}, \emph{vector.cc}, \emph{sparse.hh}, \emph{sparse.cc}, \emph{models.hh} and \emph{schemes.hh}.\\
In the \emph{vector}-files the class \emph{Vector}, which generalises the \emph{std::vector} class is built. The \emph{SparseMatrix} class construction can be found in the \emph{sparse}-files instead. Being that such files have been provided along with the assignment instructions and that their content was studied for the lessons and for Assignment 1 and Assignment 2, no discussion of them will be here reported. However, it is worth pointing out that a slight modification of the function \emph{Vector::toFile}, provided in the \emph{vector}-files has been performed. Such a modification, which is reported for clarity in the following, eases the printing of the solution to the chosen file.\\

AGGIUNGERE LST CON MODIFICA E COMMENTO A TOFILe\\

Also, in order to reduce the amount of lines printed on the terminal, one line has been commented out from the \emph{ConjugateGradient} function contained in \emph{sparse.cc}. For clarity reasons, such a line is explicitly reported in this report.\\

\begin{lstlisting}
if ( (iter % 100) == 0 ) {
	const double resi = (b - A*x).maxNorm();
	// std::cout << "Iteration " << iter << ": residual=" 	<< resi << std::endl;
	if ( resi < tolerance ) {
		return x;
	}
}
\end{lstlisting}


\subsubsection{schemes.hh file}
\label{subsubsec:schemes}
The class DIRK is made of 4 \emph{protected} variables: the matrix $A$, the vector $b$, the vector $c$ and the integer $s$ (named stages$\_$ in the code), which fully characterise the Butcher tableau of any RK method. The following instructions, defining the \emph{private} variables of the class, can be found in the schemes.hh file:

\begin{lstlisting}
protected:

	int stages_;

	std::vector<double> a_,b_,c_;
\end{lstlisting}

It may be worth pointing out that, even though \emph{a$\_$} represents a matrix, such a variable is implemented as a vector in the program.\\
The classes $FE$ (Forward Euler), $BE$ (Backward Euler) and $Heun3$ (3-stage Heun) are implemented as derived classes of $DIRK$.\\

AGGIUNGERE CLASSI IN LST\\


The function \emph{evolve}, which returns the evolved approximated solution $\mathbf{u}^{n+1}$ starting from $\mathbf{u}^n$, i.e. it evolves the solution to time $t+\tau$ from time $t$, can be found in the file as well. For flexibility reasons, the function is built as a \emph{template}. The prototype of the routine, is displayed in the following:

\begin{lstlisting}
template <class Model>
Vector evolve(const Vector &y, double t, double h, const Model &model) const // h  = tau
\end{lstlisting}

Here $y$ represents the approximated solution at the time $t$, with $h$ standing for $\tau$, i.e. the chosen discretisation stepsize.\\
As pointed out before, in case an implicit RK method is chosen, an implicit equation for $\mathbf{K}_i$ has to be solved by means of numerical methods: Newton's method has been used here.\\

INSERIRE LST CON IL NEWTON METHod\\

As it can be seen from the listed code above, the program checks if Eqn.\eqref{eqn:K_i} is an implicit equation for $\mathbf{K}_i$: if the entry $a_{ss}\neq 0$, then Newton's method is implemented.\\
The variable \emph{tol} represents the tolerance and has been here set to $10^{-6}$: the method stops once the distance (in $L_\infty$ norm) between the RHS and LHS of Eqn.\eqref{eqn:K_i} is smaller than the fixed tolerance or after a given number of iterations (which has been chosen to be equal to $10^6$) has been reached.\\

\subsubsection{models.hh file}
\label{subsubsec:models}
In this file the class \emph{HeatEquation}, whose elements completely characterise the model at hand, is defined.

INSERIRE LST CON VARIABILI PUBBLICHE \\

The class contains the variables \emph{N}, \emph{kappa} and \emph{A}.\\
A default constructor, setting the values of the variables of the class has been implemented. The other elements of \emph{HeatEquation} are the functions $f$, $df$, $T$ and $y_0$. The implementation of such functions is reported in the following.\\

INSERIRE LST CON FUNZIONI\\

The comments associated to the functions completely describe each of them, hence no additional description is required.\\
It is however worth noticing that the function \emph{T} simply returns the total integration time without performing any other task. For the sake of generality and flexibility of the code, its function structure has been preserved: as a matter of fact, such a choice eases the extension of the present code to more general cases.\\

\subsubsection{main.cc file}
The main file contains the function \emph{solve} which performs the integration of the chosen equation up to time $T$. Such a task is performed by means of repeated calls to \emph{evolve}.\\

LST CON WHILE DI SOLVE\\

As it can be seen, the function returns the solution of the whole integration. In order to store such a solution in a suitable file, the function \emph{toFile} has to be called.\\

LST CON CHIAMATA A TOFILE\\

A vector of pointers to DIRK is constructed.\\

LST COSTRUZIONE VETTORE DIRK\\

By accessing a suitable location of this vector through an appropriate choice of the input parameters, each of the RK methods can be employed. An explicit check on the consistency of the inserted parameters in implemented as well:\\

INSERIRE LISTATO CON ERROR HANDLING\\

In particular, the program checks if $N_x>1$, $k>0$ and if the scheme index has been inserted correctly, i.e. if the scheme index corresponds to a scheme implemented in the code. In the event of a wrong insertion, the program exits with a suitable warning to the user.\\

\subsection{Running the program}
The program is run through the Makefile provided.
The program accepts 3 arguments (not counting the ``./ " instruction) from the prompt (here the Makefile) and, whenever an incorrect insertion of them happens, the program exits with an appropriate warning the user.\\
The current optimisation is set to -Ofast and all following tests have been performed with this optimisation choice. Every possible error or warning has been explicitly checked by means of the instruction \emph{-Wall -Wfatal-errors -pedantic} before proceeding to performing the different tests: no error appeared. Regarding the warnings, they refer to the provided .hh files and none of them affects the correctness of the code: for these reasons, it has been chosen not to deal with them.\\
The program generates a fixed number of output files, in which the results of the tests have been stored. Some of these files are used to produce plots by means of different plotscripts.\\

\subsection{Results of the tests}
In order to check the correctness of the implementation of the code, the tests requested by the assignment have been performed. In the following, the results of the more significant tests performed will be reported and briefly discussed.\\

\subsubsection{Choice of temporal stepsize for $N_x=16$}
Having fixed a moderate spatial resolution, i.e. having fixed $N_x=16$, various temporal stepsize $\tau$ for the numerical integration have been tested. Two different value of \emph{k} have been used in order to investigate the dependence of the aforementioned tests on the thermal diffusivity. The chosen values are $k=1$ and $k=0.1$.\\
It has been shown that the choice of $\tau=10^{-3}$ guarantees the convergence of all the RK methods employed in the integration of Eqn.\ref{def:Heat}. As a matter of fact, the choice of smaller stepsizes, up to $\tau=10^{-6}$, has additionally been seen to work for the $FE$ and $Heun3$ schemes, but not for the $BE$ one, for which the choice of $\tau=10^{-4}$ does not ensure convergence.\\ 

\subsubsection{Comparing solutions over time}
As required by the assignment instructions, two different sets of values for $k$, $\tau$ and $N_x$ have been chosen in order to compare the respective solutions after one time step and after 5 different equispaced points in time later on.\\
The following choice has been made:

\begin{align}
	\label{eqn:set-1}
	k_1=, \quad\tau_1=, \quad N_{x_1}=
\end{align}

\begin{align}
\label{eqn:set-2}
k_2=, \quad\tau_2=, \quad N_{x_2}=
\end{align} 

\subsubsection{Computational performance}
16 

\section{Conclusive remarks}
In this section some additional comments and observation are presented.

\subsection{Memory}
As for the previous assignments, possible memory leaks have been checked.with the help of the software \emph{valgrind}, To do this, it is necessary to compile with \emph{-g} and \emph{-O1} optimisation. After having compiled the program, the following instruction has been used:


\begin{lstlisting}
$ valgrind --leak-check=yes ./MyProgram 16 1.0 0
\end{lstlisting}

In the present case the parameters refer to the case of $N_x=16$, $k=1.0$ and FE scheme.\\

The following output has been produced on the terminal:

\begin{verbatim}
==18980== Memcheck, a memory error detector
==18980== Copyright (C) 2002-2017, and GNU GPL'd, by Julian Seward et al.
==18980== Using Valgrind-3.13.0 and LibVEX; rerun with -h for copyright info
==18980== Command: ./MyProgram 16 1.0 0
==18980==
Finished time integration at t=0.099
==18980==
==18980== HEAP SUMMARY:
==18980==     in use at exit: 0 bytes in 0 blocks
==18980==   total heap usage: 814 allocs, 814 frees, 163,932 bytes allocated
==18980==
==18980== All heap blocks were freed -- no leaks are possible
==18980==
==18980== For counts of detected and suppressed errors, rerun with: -v
==18980== ERROR SUMMARY: 0 errors from 0 contexts (suppressed: 0 from 0)
\end{verbatim}

One can clearly see that no memory has been lost. Hence, the instruction:

\begin{lstlisting}
for (const auto& r : (schemes)) //freeing memory
{
	delete r;
}
\end{lstlisting}

which is contained in the main file, works correctly.\\

\subsection{Performances}
\label{subsec:perf}
Performances can be checked through the following commands:
\begin{verbatim}
$ valgrind --tool=callgrind ./MyProgram 16 1.0 0
$ kcachegrind
\end{verbatim}

where the parameters inserted from terminal have to be changed according to the desired model, scheme, initial step size and number eocs computed. In the present case the parameters refer o the case of $N_x=16$, $k=1.0$ and FE scheme.\\
The first command analyses the perfomances in the terms of load distrubution, while \emph{kcachegrind} is used to visualise the load distribution results.\\

Such an analysis shows that the load is well distributed among the different functions of the program.\\



\cleardoublepage
%\add1contentsline{toc}{chapter}{\bibname}
\begin{thebibliography}{99}

\bibitem{numerical-math} A. Quateroni, R. Sacco, F. Saleri;
\emph{Numerical Mathematics}, Vol.37, Springer Verlag, (2007).

\bibitem{lec-notes} T. Grafke;
\emph{Scientific Computing}, Lecture Notes, University of Warwick, (2018).







\printindex
\end{thebibliography}
\bibliography{bibliography} % BibTeX database without .bib extension
\end{document}



%----------------------------------------------------------------------------------------
%	BIBLIOGRAPHY
%----------------------------------------------------------------------------------------

%\bibliographystyle{apalike}

%\bibliography{sample}

----------------------------------------------------------------------------------------


%end{document}